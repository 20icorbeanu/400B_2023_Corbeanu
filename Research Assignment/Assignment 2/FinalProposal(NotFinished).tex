\documentclass{article}

\title{Research Proposal}
\author{Ioan-Matei Corbeanu}
\date{\today}

\begin{document}

\maketitle

\section{\textbf{Introduction}}

\begin{flushleft}

\indent Tidal Transformation of Satellites: Evolution of the internal stellar structure and dynamics of M33 and mass loss rate. As the two galaxies merge with each other, we plan to research and see how stellar evolution and mass losses of stars located in M33 changes after the collision. 

\end{flushleft}
\begin{flushleft}

\indent While we understand how stellar evolution, structure, and dynamics work it becomes tricky under extreme circumstances. One circumstance is galaxy merger. More specifically, we see more of the “rare” cases when it comes to stellar dynamics. This also helps us understand how the density of the galaxy changes as well as the evolution of the galaxy as a whole.
\end{flushleft}

\begin{flushleft}
The following is a set of 3 sources and each source has its own small detail on what we learned from it.
\end{flushleft}
\begin{flushleft}

We currently understand the general shape of the galaxy and the types of stars that exist in certain locations. In specific, the stars at the center of the galaxy tend to be newer and are richer in metallicity, while stars on the outer rings are older and have metal-poor. Over time, the data showed that star formation in the inner rings started before it did in the outer. 
	
\noindent \textbf{2. Specific research question:}
	
\indent Stars: Evolution of the internal stellar structure and dynamics of M33 and mass loss rate

\end{document}